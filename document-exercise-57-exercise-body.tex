% ------------------------------------------------------------------------
% file `document-exercise-57-exercise-body.tex'
%
%     exercise of type `exercise' with id `57'
%
% generated by the `exercise' environment of the
%   `xsim' package v0.11 (2018/02/12)
% from source `document' on 2018/08/08 on line 167
% ------------------------------------------------------------------------
En bil starter fra ro (startfart lik 0) og har akselerasjon på 3 m/s2. Lag et program som for hvert sekund fra 0 til 10 sekunder regner ut farten til bilen  og hvor langt den har gått, og skriver det ut til skjerm. Hint: farten v og strekningen s ved tid t er gitt ved v = a*t og s = 0.5*a*t**2 der a = 3 er akselerasjonen.

Utskriften skal se omtrent slik ut:
\begin{tabular}{l|r|l|r|l|r}
tid: & 0.00 s  &  fart: & 0.00 m/s &   strekning: &  0.00 m\\
tid: & 1.00 s  &  fart: & 3.00 m/s &   strekning: &  1.50 m\\
\ldots & & & & &
\end{tabular}
