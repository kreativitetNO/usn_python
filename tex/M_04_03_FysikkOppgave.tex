######################################################################## 
KAP. 4.3  SYMPY: FYSIKKOPPGAVE  (optional)

(Litt vanskelig. Kanskje spare til en fysikktime.) 
En kloss med masse m ligger på et flatt bord. 
To snorer er festet i klossen.
Den ene er horisontal og drar klossen mot høyre med kraft Sx.
Den andre er vertikal og drar klossen rett opp med kraft Sy.
Der er friksjon mellom klossen og underlaget.
Friksjonskraften er R = mu*N, horisontal, i motsatt retning av Sx.
(N er normalkraften, mu er friksjonstallet.) 
Anta at Sy ikke er stor nok til at klossen løfter fra bakken, der er altså ingen akselerasjon i y-retning.
Anta at Sx er stor nok til at klossen får en akselerasjon a mot høyre. 
Finn akselerasjonen som funksjon av variablene m, Sx, Sy og mu. 
(Tegn figur.) 
Vi har fire ligninger:
G = m*g       # m:masse, g:tyngdeakselerasjonen (9.81 m/s**2)
R = mu*N      
Sy + N = G    # y-retning: Newtons 1. lov: summen av kreftene er null (fordi der ikke er akselerasjon i y-retning)
Sx - R = ma

Vi må skrive ligningene om så vi får 0 på høyresiden. 
Da blir det slik:

from sympy import *
m,g,mu,Sx,Sy,N,R,G,a = symbols('m g mu Sx Sy N R G a')
ans, = linsolve( [Sx-R-m*a, Sy+N-G, R-mu*N, G-m*g], (a,R,N,G) )

Rett svar for akselerasjonen er da
ans[0]        # (Sx + Sy*mu - g*m*mu)/m

Vi kan sette inn noen tall, f.eks. Sx = 10, Sy = 2 (enhet er Newton for begge, men vi skriver det ikke inn).
Og mu = 0.3, m = 1 (kg), og g = 9.81 (m/s**2)
ans.subs({Sx:10, Sy:2, mu:0.3, m:1, g:9.81})   # gir (7.657, 2.343, 7.81, 9.81)
altså en akselerasjon på 7.657 m/s**2.



Over satte vi inn 4 ligninger og hadde 4 variabler.
Vi kunne forenklet litt først, og fått to ligninger,
Sy + N - m*g = 0
Sx - mu*N - m*a = 0
Disse kunne vi da løst for a og N:
linsolve( [Sx - mu*N - m*a, Sy + N - m*g], (a,N) )
Resultatet blir det samme. 
