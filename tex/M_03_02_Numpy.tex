######################################################################## 
KAP. 3.2  NUMPY: array

Numpy står for Numerical Python og inneholder matematiske ressurser 
som gjør det enklere å bruke Python til numeriske beregninger, plotting etc. 

(Scipy (Scientific Python) er en lignende pakke du fort kan komme borti. 
Den bygger videre på numpy.)

Spesielt er det datatypen array (ndarray, n-dimensional array) vi trenger fra numpy. 
(Datatyper vi har vært borti så langt: int, float, str, bool, list, tuple, (dict).) 
Et array kan lages fra en liste:
import numpy
mylist = [1, 2, 3.3, 4.5, 4] 
myarr = numpy.array(mylist) 
type(myarr)    # (numpy.ndarray)
print(myarr)

(NB: når du printer et array, vises kommaet mellom elementene.)  

Et array er som en spesiell type liste. 
Du kan endre verdier, f.eks.
myarr[1] = 2.44
Men du kan ikke endre lengden (legge til eller fjerne verdier) ved append(), 
som vi kan gjøre med en vanlig liste. 
Skriv myarr. og trykk TAB for å se hvilke funksjoner denne datatypen tilbyr. 
(Vi skal bruke bare noen få av disse.) 


Andre måter å lage et array på, er:
arr1 = numpy.zeros(6)    # lager et array med 6 nuller
arr2 = numpy.ones(6)     # lager et array med 6 enere

Der finnes to funksjoner for å lage et array som øker jevnt mellom en minste og en største verdi. 
Den ene: arange() er nesten det range() er for lister: 
arr3 = numpy.arange(1, 2, 0.1) 
print(arr3)
Her lages et array med verdier 1., 1.1, 1.2, ..., 1.9  (NB: maksverdien 2 er ikke med)
Til forskjell fra range(), tillater arange() desimalverdier. 

Den andre: linspace(a,b,n) lager et array med n verdier. Første verdien er a, siste er b. 
arr4 = numpy.linspace(1,2,11)  # gir 1., 1.1, ..., 2.0  
print(arr4)


Metoden med arange() eller linspace() er praktisk når vi skal lage figurer. 
Typisk vil vi ønske en jevn sekvens av x-verdier, som er det linspace() og arange() gir.  
Så regner vi ut y-verdier, en for hver x-verdi, utifra en prosedyre eller formel, 
og plotter til slutt x mot y. 

Ex.:
import numpy
x = numpy.linspace(0,1,11)
y = numpy.zeros(11)
for i in range(11): 
    y[i] = x[i]**2
print(x)
print(y)

Vi bruker her en loop for å fylle y-arrayet.
I for-loopen er i først lik 0 og vi regner ut y[0] basert på x[0], så blir i lik 1 etc., 
frem til i=10, som er siste elementet i x og y. 

Dette er fint, og vi er nå forsåvidt klare til å bruke matplotlib for å tegne figuren.


Men først: vektorisering
Der finnes imidlertid en raskere og kortere (og dermed bedre) måte å regne ut y på.
I tilfellet ovenfor kan loopen erstattes av vektorisering. 
Vi kan ganske enkelt skrive:
y = x**2
Python vil da se at x er et array og internt gjøre det vi gjør med for-loopen, 
men på en mer effektiv måte.
Vi trenger altså ikke for-loopen vi hadde i eksempelet ovenfor. 
(NB: dette ville ikke fungert dersom x var en liste av tall (x = [0,.1,.2,.3]), 
vektorisering fungerer kun med arrays.) 



Programmet blir da:
import numpy
x = numpy.linspace(0,1,11)
y = x**2
print(x)
print(y)

Vi kan også printe ut resultatet slik:
for i in range(len(x)):
    print("x:%5.3f  y:%5.3f" %(x[i], y[i]))


For å tillate vektorisering
Numpy inneholder egne versjoner av forskjellige funksjoner for bruk sammen med array 
der de opprinnelige funksjonene ikke kan brukes til vektorisering. 

F.eks. kan vi skape et array av 10 tilfeldige desimaltall mellom 3 og 5 med numpys 
egen numpy.random.uniform(): 
arr10 = numpy.random.uniform(3,5,10)

Dette er ikke mulig med den vanlige random.uniform(). 
Den tar bare to argumenter, og gir ett tilfeldig tall. 
Skulle vi laget en liste med ti tilfeldige tall, måtte vi brukt en for-loop. 
list10 = []
for i in range(10): 
    list10.append( random.uniform(3,5) ) 


Og dersom vi har et array og vi vil ta kvadratroten av hvert element i arrayet, 
kan vi gjøre: 
arr10b = numpy.sqrt(arr10)
Igjen: Dette er ikke mulig med math sin sqrt-funksjon. 
Prøv: 
import math 
arr10c = math.sqrt(arr10). 
Du får feilmelding TypeError. 


Så: Med vektorisering gjør numpy at vi kan utføre matematiske operasjoner 
som vi ellers måtte brukt en loop for å gjøre. 
Det er effektivt (raskt), og gir også mer konsis kode. 


NB: Når du printer eller viser et array i spyder, vil alle elementene vises dersom arrayet 
har ti eller færre elementer.
Er det flere enn ti, vil kun de 3 første og de 3 siste elementene vises, med '...' imellom.
Prøv: 
numpy.random.uniform(0,1,10)
numpy.random.uniform(0,1,11)



Oppgave 1: Lag et array som går fra 0 til 10 (inkludert endepunktene). 
Steglengden (spacing) skal være 1  (dvs. [0, 1, 2, ..., 10]). 


Oppgave 2: Lag et array som går fra 0 til 10 (inkludert endepunktene). 
Steglengden (spacing) skal være 0.2  (dvs. [0, 0.2, 0.4, ..., 9.8, 10])


Oppgave 3: Sammenlign numpy.linspace(0,10,6) med numpy.linspace(0,10,5). Forstå forskjellen


Oppgave 4: Lag et array med 10 tilfeldige desimaltall mellom 0 og 100 
uten å bruke numpy's egne random-funksjoner. 
Dvs. du må bruke random.uniform(0,100).
Gjør så det samme ved hjelp av numpys randomfunksjoner (numpy.random.uniform).


Oppgave 5: 
En bil starter fra ro (startfart lik 0) og har akselerasjon på 3 m/s2. 
Lag et program som for hvert sekund fra 0 til 10 sekunder regner ut farten til bilen 
og hvor langt den har gått, og skriver det ut til skjerm. 
Hint: farten v og strekningen s ved tid t er gitt ved v = a*t og s = 0.5*a*t**2 der a = 3 er akselerasjonen.
Utskriften skal se omtrent slik ut:
tid:  0.00 s    fart:  0.00 m/s    strekning:   0.00 m
tid:  1.00 s    fart:  3.00 m/s    strekning:   1.50 m
... etc. 


Lese mer: https://en.wikipedia.org/wiki/NumPy
