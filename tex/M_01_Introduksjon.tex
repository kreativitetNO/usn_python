\chapter{Introduksjon}
\section{Installering av Anaconda}
\begin{enumerate}
\item Gå til \url{https://www.anaconda.com/download/}
\item Klikk Windows, Mac eller Linux nederst på skjermen.
\item Klikk på Python 3.6 version  DOWNLOAD  (64-Bit).
\item Når filen har lastet ned (typisk 400 MB), kjører du den.
\item Under installasjon er det greieste å si OK til alt.
\end{enumerate}
{\em NB: Under windows kan det oppstå problemer om Anaconda installeres i en path med mellomrom, f.eks. \texttt{PROGRAM FILES}. Dette skal ikke være et problem om du følger det installasjonen selv foreslår.}

Gjør oppgaver for å lære, ikke bare for å gjøre de. Noen delkapitler har litt for mange oppgaver. Når du føler oppgavene ikke får deg til å lære noe nytt, er det kanskje på tide å gå videre til neste oppgave eller neste underkapittel. 

Mange av delkapitlene er markert med optional. De rekker vi sannsynligvis ikke, men det er alltids tillatt å ta en titt. 

Dette dokumentet gir en grei innføring i de mest grunnleggende delene av python,  samt lett overfladisk beskrivelse av noen ekstrapakker (numpy, matplotlib, pygame). 

Der er alltid (mye) mer å lære om et emne: søk på nettet når du lurer. Særlig stackoverflow.com er bra for å lete etter svar på spørsmål av typen "how to ... in python". 

Noen filer (inkludert denne) er tilgjengelige på \url{http://kilelabs.no/p/}


\section{Hva er programmering?}
(Under arbeid)