######################################################################## 
KAP. 6.2  PYGAME: Introduksjon til skjerm, sprite, lyd


import pygame   # required
Det kan være lurt å importere noen konstanter etc.: 
from pygame.locals import *     # optional 
Så må pygame initialiseres
pygame.init()   # required

# Variabler 
# Det er nesten alltid bedre å bruke variabler 
# enn å skrive inn tallene direkte i funksjoner etc. 
WIDTH  = 800  # skjermflaten 
HEIGHT = 600  # skjermflaten

# Farger oppgis med tre tall, hvert mellom 0 og 255
# Tallene står for mengde Rødt, Grønt og Blått (RGB)
# Det kan være greit å definere noen så: 
DARKGREEN = ( 0,  100 ,  0 )
RED = (255,0,0)
BGCOLOR = DARKGREEN


# Lage "spill-vinduet" 
skjerm = pygame.display.set_mode((WIDTH, HEIGHT))
# Nå kom vinduet opp. Svart.  
# (Vi har ikke full kontroll på hvor det plasserer seg.) 

# Vi kan alltids gi vinduet en tittel:
pygame.display.set_caption("Leking er beste form for læring") 

# La oss gi vinduet en farge:
skjerm.fill(DARKGREEN)
# Men ingenting skjer. 
# Vi trenger følgende linje for at endringer vi har gjort skal vises. 
pygame.display.flip()


Sprite 
# Vi vil putte en ball (et grafisk objekt, en sprite) på skjermen 
# Først må vi laste inn fra fil (png, jpg, gif, ..) 
# Kan finne den her:  http://kilelabs.no/p/tut/ball1.png 
ball = pygame.image.load("ball1.png")

# Det er lurt (men ikke absolutt nødvendig) å konvertere grafikken
ball = ball.convert_alpha() 
# Hver sprite har et rektangel rundt seg, det trenger vi 
ballrect = ball.get_rect()

# Prøv
print(ballrect)    # (eller bare skriv ballrect)
# Du får ut <rect(0, 0, 95, 95)>
# Det er x- og y-pos, samt bredde og høyde
# Prøv ballrect. og trykk TAB
# Da ser du hva ballrect inneholder, hva du kan gjøre med den
# F.eks.: angir ballrect.x  x-posisjonen 
# Prøv:
ballrect.x = 45
print(ballrect.x)   # 45
print(ballrect.y)   # 0
# Ingenting skjer, men du har nå endret x-posisjonen. 

# La oss få ballen opp på skjermen:
# Det gjøres slik, med blit-funksjonen
skjerm.blit(ball, ballrect)
# Første argument er ballen, det andre er ballrektangelet
# Egentlig er det nok å angi posisjonen (x,y) som andre argument.
# Så skjerm.blit(ball, (130,200)) hadde også fungert. 
# Som tidligere er det ingenting som skjer før du gjør 
pygame.display.flip()
# Nå kom ballen på skjermen.
# Stemmer størrelsene?
# Ballen er 95x95, skjermen du satte opp er 800x600.
# Det ser ut til å stemme.

# Posisjonen ser også ut til å stemme.
# Men vi ser at y=0 betyr øverst. y-posisjonen vokser nedover på skjermen. 


# Du kan flytte ballen ved å endre posisjonen
ballrect.x = 100
# Så må du blitte ballen på skjermen 
skjerm.blit(ball, ballrect)
# Og til slutt må du oppdatere skjermen 
pygame.display.flip()

# Men det var kanskje ikke helt slik vi tenkte det.
# Det forrige bildet av ballen forsvant ikke, ble bare delvis overskrevet.

# Datagrafikk fungerer ikke helt som vi kanskje ser for oss. 
# Det er ikke egentlig objekter (f.eks. baller) som beveger seg omkring. 
# Det er bilde på bilde som vises etter hverandre. 
# Og hvert bilde må tegnes fra grunnen.
# Vi kan forsåvidt ta utgangspunkt i det forrige bildet,
# men må da fjerne de delene som har flyttet seg,
# tegne på bakgrunnen der de var, så tegne objektene på sine nye plasser.
# (Det kan bli komplisert, men er ofte nødvendig når du behov for rask
# oppdatering av skjermen.)

# Vi trenger altså å gjøre følgende:
skjerm.fill(DARKGREEN)       # grønn bakgrunn
skjerm.blit(ball, ballrect)  # blitte ballen på skjermen
pygame.display.flip()        # først nå vises endringene

# Hvorfor vises endringene først når vi gjør flip() ?
# Skjermen har egentlig en kopi av seg selv.
# Dvs. der er to buffere som brukes.
# Den ene er den aktive, den som viser skjermbildet.
# Den andre (den inaktive) blir satt lik den aktive like etter flip().
# (Den blir en kopi av den nye aktive.) 
# Når vi så gjør ting som skjrem.fill() eller skjerm.blit(),
# så er det den inaktive vi gjør endringer på.
# Når vi gjør flip(), så flipper vi mellom de to bufferene. 
# Da blir den inaktive den aktive (og skjermbildet endres), 
# og den som var den aktive tar en kopi av den nye aktive
# og blir selv den inaktive.
# 
# Grunnen til at det er slik, er at det å endre skjermbildet
# er noe av det som koster mest, noe av det mest intensive. 


# La oss lage en ball til 
ball2 = ball.copy()  # Vi baserer den på den forrige
# La oss skalere den til størrelse 50x70 (fra 95x95)
ball2 = pygame.transform.scale(ball2, (50,70))
ball2rect = ball2.get_rect()
ball2rect.y = 100
# La oss få den på skjermen
# Denne gangen vil vi bare ha den nye ballen på uten å endre det gamle,
# så da er det nok å gjøre
skjerm.blit(ball2,ball2rect)
pygame.display.flip()


ball3 = pygame.transform.rotate(ball, 180)  # rotere 180 grader
ball3rect = ball3.get_rect()
ball3rect.x = 200

# Sette ballen på skjermen.
# Igjen bryr vi oss ikke med å fjerne og tegne på resten
# (for det har ikke bevegd seg)
skjerm.blit(ball3,ball3rect)
pygame.display.flip()
# Ikke så lett å se umiddelbart, men ballen er rotert 180 grader



TEKST PÅ SKJERMEN
Det er litt omstendelig å putte tekst på skjermen.
Først må vi lage et font-objekt (tekst-objekt):
font = pygame.font.Font(None, 36)
Første argumentet er fonttypen. Med None vil pygame velge for deg.
Problemet med å angi fonter, er at de ikke nødvendigvis finnes
der du er. De må være installert for å kunne brukes.
Andre arguementet angir fontstørrelsen.

Så må en surface (et grafikk-objekt) lages 
tekst = font.render("Python rules!", 1, (10, 10, 10))
Første argumentet er tekststrengen.
Andre argumentet er 1 (True) for antialiasing, som gjør overgangen
mellom tekst og bakgrunn mindre hard, eller 0 (False) for ingen antialiasing.
Tredje argument er fontfargen i (R,G,B).
Man kan også ha et fjerde argument, det er da fargen til tekstbakgrunnen.

Så er det bare å blitte teksten til skjerm, samt flippe skjermen:
skjerm.blit(tekst, (50,400) ) 
pygame.display.flip()
Som tidligere angir andre argumentet til blit() posisjonen til
øvre-venstre hjørne av objektet som du blitter (her tekst). 


En til, rød tekst på blå bakgrunn (husk RGB): 
tekst2 = font.render("pygame rules!", 1, (255,0,0), (0,0,255))
skjerm.blit(tekst2, (50,430) ) 
pygame.display.flip()

Vi kan sette samme teksten flere steder
skjerm.blit(tekst2, (0,0))
pygame.display.flip()

Noen ganger trenger du å vite størrelsen på teksten for å kunne plassere
den slik du vil.
Størrelsen får du med
tekst2.get_width()
tekst2.get_height()
eller
tekst2.get_size()
som gir en tuple med vidde og høyde. 

(Eks.: Hvordan skal du da plassere teksten sentrert på skjermen?
Du må da også vite skjermstørrelsen. Det gjør du som oftest.
Eller du kan bruke skjerm.get_width() og skjerm.get_height()
Så må du regne ut hvor øvre-venstre hjørne av teksten skal være.
x blir  skjerm.get_width()/2 - tekst2.get_width()/2
Og tilsvarende for y.) 


Fonten kan også gjøres bold, italic etc.
Skriv font. og trykk TAB og se mulighetene
Prøv:
font.set_italic(True)
tekst3 = font.render("pygame rules!", 1, RED)
skjerm.blit(tekst3, (100,200))


FONTER
Fonter kan du få tak i på maskinen din.
Der er flere metoder.
F.eks. vil pygame.font.get_fonts()
gi deg en liste over hvilke systemfonter du har.
Anta at du har 'arial' i den lista.
Da kan du hente den via SysFont()
font_arial = pygame.font.SysFont('arial', 36)

Det er også mulig å finne fonter via 
matplotlib.font_manager.FontManager()

Og det er mulig å definere en font i en ttf-fil
som da må være tilgjengelig for python. 



LYD
import pygame
pygame.init()

pygame har to måter å gjengi lyd på.
Den ene, Sound(), er ment for (korte) lydeffekter.
Flere kan spille samtidig. 
Den andre, music(), er ment for (bakgrunns)musikk.
Her går det ikke å spille flere samtidig.
Men man kan naturligvis spille Sound() mens music() pågår.

LYD SOM LYDEFFEKT: Sound(): 

lyd1 = pygame.mixer.Sound("plopp5.ogg")
# Du kan finne lyden her:
# http://kilelabs.no/p/spaceinvaders/spacedata/plopp5.ogg
# Andre formater WAV. 

lyd1.  pluss TAB viser følgende funksjoner:
lyd1.get_length()     # lengden i sekunder
lyd1.play()           # start
lyd1.stop()           # stop
lyd1.fadeout(4000)    # i millisekunder (her: 4s)
lyd1.get_volume()     #
lyd1.set_volume(1.0)  # mellom 0.0 og 1.0

Det går an å spille en og samme lyd(objekt) flere ganger samtidig.
Det vil være aktuelt hvis du f.eks. har flere baller som spretter omkring,
eller hvis du har en mengde aliens som skyter.
lyd1.play()
lyd1.play()
lyd1.play()
lyd1.stop()   # vil i så fall stoppe alle lyd1-lyder som ikke allerede er ferdig. 

Er du interessert i vite hvor mange som fortsatt spiller: 
lyd1.get_num_channels()



LYD SOM (BAKGRUNNS)MUSIKK: music()

pygame.mixer.music.load("piano1.ogg")
# http://kilelabs.no/p/spaceinvaders/spacedata/piano1.ogg 

pygame.mixer.music.  pluss TAB for å se hva som er mulig 
pygame.mixer.music.play()    # spille av en gang
pygame.mixer.music.play(3)   # spille av 1+3 ganger
pygame.mixer.music.play(-1)  # spille av uendelig mange ganger

Interessante funksjoner:
get_pos()     # gir posisjonen i millisekunder
set_pos(3.5)  # angir posisjon å spille fra (i sekunder)
play(0,3.5)   # starter avspillingen etter 3.5 sekunder 
pause()
unpause()

