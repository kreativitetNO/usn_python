######################################################################## 
KAP. 3.3  MATPLOTLIB

I pakken matplotlib er der mange ressurser som har med plotting å gjøre.
Vi skal her bruke en underpakke (sub-package) som heter pyplot.
(Den er inspirert av plottingen i Matlab.)

Nedenfor skiller vi mellom (figur)vindu, figur og graf. 
  Vindu: selve figurvinduet, som du kan flytte rundt på skjermen om du vil 
  Figur: Koordinatsystem
  Graf:  Selve linja eller punktene som representerer funksjonen du plotter

Skriv inn følgende program og lagre det som tut_matplotlib1.py. 

import numpy 
import matplotlib.pyplot as plt 
x = numpy.linspace(0,1,11)
y = x**2
plt.xlabel('x')   # Vi kan sette titler på aksene
plt.ylabel('y')   # y-aksen
plt.title('En fin funksjon')    # figurtittel 
plt.plot(x,y,'*--') 
  # x er array med x-verdiene
  # y er array med y-verdiene
  # neste argument pynter på grafen: b er blå, * er sirkel, -- er stiplet linje
plt.show()   

Siste linja er kun nødvendig dersom figuren vises i eget vindu. 
Viser du grafen inline, (som er det spyder i utgangspunktet gjør), 
gjør linja ingenting (og skader dermed heller ikke). 

Kjør programmet. 
Der er to måter figuren kan vises.
Enten skjer det inline i IPython-konsollen, som er default. 
Eller så blir et nytt vindu åpnet ved kommandoen ovenfor. 
I inline-modus har siste linja, plt.show() ingen effekt. 
Men er du ikke i inline-modus, og et nytt vindu åpnes, må den linja være med 
ellers lukker vinduet seg når programmet er ferdig, dvs. så å si umiddelbart.  
I sistnevnte tilfelle kan vi ikke bruke IPython-terminalen så lenge vinduet er oppe. 
Avslutt i så fall ved å klikke krysset oppe i høyre hjørne på figur-vinduet. 
Men du er sannsynligvis i inline-modus. 

For å endre fra/til inline plotting kan du enten ...:
klikke fra toppmenyen: 
Tools / Preferences / IPython console / Graphics
og velg så Backend (Inline eller Automatic) 
Automatic gir et eget figur-vindu.
Du må muligens restarte IPython-konsollen: Consoles / Restart kernel
Det er tar et par sekunder.

... eller du kan skrive i IPython-konsollen en av følgende: 
%matplotlib inline   # figur inline i IPython-konsollen
%matplotlib qt       # figur i eget vindu


En annen mulighet for å tvinge figurer til å lages i eget vindu er å bruke 
Python-konsollen i stedet for IPython-konsollen. 

(Så: dersom du har problem med å få det som du vil, prøv en av de andre mulighetene.) 



Når du plotter i en eget vindu i IPython-terminalen, trenger du plt.show() 
for å unngå at programmet avsluttes og vinduet forsvinner med en gang. 
Dette låser imidlertid terminalen, slik at du må klikke vekk figurvinduet 
for å igjen kunne skrive i terminalen. 
Dette er ikke alltid det du ønsker. 
Og der er en løsning. 
Det er mulig å plotte såkalt interaktivt. 
Figuren kommer da opp allerede med plt.plot(), og forsvinner ikke når programmet er ferdig, 
og du trenger ikke plt.show(). 
Du kan da gjøre endringer. Terminalen låser seg ikke. 
Det du må gjøre er å bruke plt.ion(), der ion betyr interactive on. 
(Interaktiv plotting slås av ved plt.ioff()) 


# Prøv (i IPython-konsollen)
import numpy 
import matplotlib.pyplot as plt 

x = numpy.linspace(0,1,11)
y = x**2

plt.ion()              # interactive on 
plt.plot(x,y,'b*--')   # Figuren vises med en gang, du trenger ikke plt.show()
plt.xlabel('X')        # Kan legge til ting underveis
plt.ylabel('Y') 

Du kan lagre figuren: 
plt.savefig('x_i_andre.png')   # Tillatte formater: png, jpg, pdf, ... 


Skal du tegne en ny graf, må du lukke figurvinduet eller skrive plt.cla(), 
ellers blir grafen plottet i den samme figuren. 



Oppgave 1: Tegn grafen f(x) = x**3-x for -2 < x < 2. 
Sett passende label på x- og y-aksen, og gi grafen en tittel.
Tegn grafen både inline og i eget vindu (i to ulike kjøringer). 



Mye mer å lese (matplotlib inneholder mye): 
http://www.labri.fr/perso/nrougier/teaching/matplotlib/  (tutorial) 
matplotlib.org
http://matplotlib.org/gallery.html 
http://matplotlib.org/examples/index.html
http://matplotlib.org/api/pyplot_summary.html
http://matplotlib.org/resources/index.html  (Matplotlib-tutorials etc.)
