######################################################################## 
KAP. 4.2  SYMBOLSK REGNING, SYMPY  (optional)

Python har en modul som gjør det mulig å regne symbolsk: sympy
Prøv i IPython-konsollen:
import sympy
Skriv så sympy. og trykk TAB
Du ser da alle funksjoner som finnes i pakken. Det er mange.

For å bruke en variabel symbolsk må du fortelle python at den skal være en symbolsk variabel:

from sympy import *
x = Symbol('x')   # kan også bruke symbols(), se lenger ned 
type(x)           # Er av type sympy.core.symbol.Symbol

Vi kan lage et uttrykk f
f = x**2 - 2*x - 3 
print(f)

Vi kan løse ligningen x**2 - 2x - 3 = 0
solve(f)
Svaret er [-1, 3]. Der er altså to løsninger, x=-1 og x=3.
Som vi husker, siden det er en andregradsligning.

Vi kan også faktorisere uttrykket
factor(utt1)    # gir (x-3)*(x+1)

Hvis vi ser på utt1 som en funksjon f(x), husker vi fra matematikken at f(x) har nullpunkter for x=3 og x-1. 
Hvis x blir veldig stor eller veldig liten, vil f(x) bli veldig stor. 
Et sted vil der dermed være et minimumspunkt.
Minumuspunktet er der den deriverte til funksjonen er lik null. (I dette tilfellet.) 
Kan python finne den deriverte? 
f_der = diff(f,x)    # gir 2x-2
NB: siden der kun er en symbolsk variabel (x) i f(x), ville også f_der = diff(x) fungert. 

Vi kan erstatte x med et annet symbol eller med et tall:
f.subs(x,4)    # som gir svaret 5
NB: I linjen over blir ikke f erstattet med tallet, f er fortsatt symbolsk: 
print(f)
(f = f.subs(x,4) ville erstattet f)

Vi kan ha uttrykk med flere variable: 
y,a = sympy.symbols('y a')    # NB: med symbols() kan flere symbolske variable defineres med ett kall 

g = 2*x*y - 3*a*x**2

Prøv:
print(g)
solve(g,x)        # Løser med hensyn til x. Gir to løsninger. 
simplify(g)       # Prøver å forenkle. I dette tilfellet velger python å faktorisere. 

g2 = g.subs(a,3)
g3 = g2.subs(y,x)
Dersom du vil erstatte flere verdier samtidig, må du bruke en dict: 
g4 = g.subs({a:3, y:x})
(Vi kan også bruke dict som argument når bare en verdi skal erstattes, g2 = g.subs({a:3}), men det er mer å skrive.) 


Vi kan derivere g med hensyn til en av variablene, f.eks. x:
diff(g,x)


Python kan også symbolsk integrasjon:
F = integrate(f,x) 
print(F)
integrate(g,x)




Løse ligningssett 
Hvordan løse et ligningssett bestående av to ligninger.
Eks: x+y=1 og x-y=-3
(To variable, x og y, og to ligninger, da kan vi finne en løsning.)
Metoden er å skrive ligningene slik at høyresiden er 0.
Da blir det:
x + y - 1 = 0  og  x - y + 3 = 0
Kan løses slik:
lign1 = x + y - 1
lign2 = x - y + 3
from sympy.solvers.solveset import linsolve
linsolve( [lign1, lign2], (x,y) )     # Det andre argumentet (x,y) angir hvilke variabler vi skal løse for
Gir svaret {(-1,2)} som betyr at x = -1 og y = -2.
(Sjekk manuelt ved å legge sammen de to ligningene, og trekke den ene fra den andre.) 

Kan også skrive det eksplisitt:
linsolve( [x + y - 1, x - y + 3], (x,y))

Kan også ha med flere symbolske variable: 
linsolve( [x + y - a, x - y + 3], (x,y))

Da kunne det også gitt mening å løse for f.eks. x og a:
linsolve( [x + y - a, x - y + 3], (x,a))



Vi ønsker gjerne å lagre svaret i en variabel:
ans = linsolve( [lign1, lign2], (x,y) )  
print(ans)
Hvordan får vi tak i svaret?
type(ans)     # sympy.sets.sets.FiniteSet()

Det enkleste er slik:
ans, = ans    # Merk kommaet
Da er ans2 en tuple, og kan akssesserers med indekser,
print(ans2[0])  # f.eks. slik
x1,y1 = ans2    # eller slik 

Aller enklest er å bruke komma-trikset direkte i linsolve-kallet: 
ans, = linsolve( [lign1, lign2], (x,y) )  



Oppgave 1: Finn nullpunktene for funksjonen f = 3*x**3 + 82.5*x**2 - 640.5*x + 765 = 0


Oppgave 2: Finn x fra ligningen   1/(x-a) = 2/(x+a)


Oppgave 3: Løs ligningssettet 2x + a = x + y + 5  og  3x - y = 2a - 3  med hensyn på x og y (for variabel a).
Finn også løsningene når a = 5. 


