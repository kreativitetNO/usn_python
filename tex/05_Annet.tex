\chapter{Tillegg 1}

\section{Mer om lister, dicts og løkker}

Å kopiere en liste:

\begin{usncodebox}
L1 = ['a','c','b']
L2 = L1
\end{usncodebox}

Dette lager ikke en kopi av liste L1. L2 peker bare på samme objekt i hukommelsen. Så dersom den ene endres, endres også den andre, det er samme objekt. Se:
\begin{usncodebox} 
print(L2)       # L2 er ['a','c','b']
L1.append('d')  # Vi endrer L1
print(L2)       # L2 har blitt ['a','c','b','d']
\end{usncodebox}

Dersom vi vil lage en frittstående kopi, kan vi bruke :
\begin{usncodebox}
L3 = L1.copy() # (eller L3 = list(L1))
\end{usncodebox}

Å kopiere en dictionary: 

Hvis du vil lage en frittstående kopi av en dictionary, må du bruke copy()
\begin{usncodebox}
D1 = {'en':'one', 'to':'two'}
D2 = D1           # peker på samme objekt
D3 = D1.copy()    # lager en nytt, frittstående dictionary
\end{usncodebox}

To måter å gå gjennom en liste på: 
Via indeks:
\begin{usncodebox}
for i in range(len(L1)):
    print("%i  %s" %(i,L1[i]))
# Eller bare via verdiene: 
for L in L1:
    print(L) 
\end{usncodebox}

Sortere en liste (ordne alfabetisk m.m.)
\begin{usncodebox}
sorted(L1)               # endrer ikke L1, men gir en sortert liste 
L1s = sorted(L1)         # 
L1.sort()                # sorterer (endrer) L1
L1.sort(reverse=True)    # alfabetisk baklengs
sorted(L1,reverse=True)  # endrer ikke L1
\end{usncodebox}

Løkker: break og continue 

Continue brukes for å droppe resten av linjene og gå direkte til neste iterasjon i for- eller while-loopen
Eksempel:
\begin{usncodebox}
for i in range(5): 
    print(i, 'before')
    if i == 2: continue
    print(i, 'after')
\end{usncodebox}

Break brukes for å hoppe helt ut av en loop: 
Eksempel 1: 
\begin{usncodebox}
for i in range(5): 
    print(i)
    if i == 2: break
\end{usncodebox}

Eksempel 2
\begin{usncodebox}
for i in range(3): 
    for j in range(5): 
         if j == 2: break  # hopper ut av j-loopen
         print(i,j)
\end{usncodebox}

\section{Klasser (utgår)}

\section{Nyttige småtterier}

\section{tkinter: Lage GUI (optional)}

Tkinter er den vanligste pakken for å lage menyer, knapper etc (men kanskje ikke så moderne). 

For mer (veldig mye) info: 
Tkinter Basics - Python Programming Tutorials:
\url{https://docs.python.org/3/library/tk.html}

Minimalt program som viser frem label, input, text og funksjon
\url{http://kilelabs.no/p/tut/tut_tkinter_menu.py}

Skrått-kast-kalkulator: 
\url{http://kilelabs.no/p/tut/tut_tkinter_skraattkast1.py}

Skrått-kast-kalkulator med figur: 
\url{http://kilelabs.no/p/tut/tut_tkinter_skraattkast2.py}
