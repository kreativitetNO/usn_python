######################################################################## 
KAP. 3.6  ANIMASJON i for-løkke  (optional)

Ting som beveger seg er gøy, og kan også være informativt. 
Det er mulig å lage ting som beveger seg via matplotlib.
Det kan være funksjoner som oppdaterer seg ettersom tiden går, f.eks. sin(t).
Det kan være (veldig) enkle fysikkanimasjoner, som f.eks. en ball som blir kastet.
Animasjonen kan vises mens programmet kjører, eller man kan også lage en liten
video-fil (mp4) som f.eks. kan legges ut på nettet.
Det kan imidlertid også bli litt vel teknisk på dette nivået. 


En enkel, men begrenset og vanligvis hakkete måte er å oppdatere grafen i en loop. 
Det fungerer ikke dersom figurene vises inline.
Man bruker da ingen av de mer avanserte animasjons-funksjonene i matplotlib,
og kan heller ikke lage mp4-fil.

Eksempel (skrått kast):
(Skriv inn og lagre som tut_skraattkast_anim_for.py) 

import numpy as np
from matplotlib import pyplot as plt

plt.ion()   # interaktiv kjøring
fig = plt.figure()    # åpner et figurvindu
akse = plt.axes(xlim=(0, 25), ylim=(0, 10))  # setter opp koordinatsystem

# initialisere kastet
vx0 = 10       # startfart i x-retning
vy0 = 10       # startfart i y-retning

dt = 0.1       # oppdaterer hvert 0.1 sekund
t = 0          # starttid
ay = -9.81     # akselerasjon i y-retning er tyngdeakselerasjonen (nedover, derfor minus)
x = 0 
y = 0          # startposisjon 

while y >= 0:  # stopper når ballen når bakken
    x = vx0*t
    y = vy0*t + 0.5*ay*t**2
    akse.scatter([x],[y],c='r')  # her plottes den nye posisjonen (den forrige slettes ikke)
    t = t + dt          # oppdatere tiden
    plt.pause(0.02)     # liten pause for aa gjoere plottingen litt realistisk



Dette fungerer forsåvidt, men det kan altså gjøres bedre.



