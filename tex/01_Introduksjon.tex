\chapter{Introduksjon}
\section{Installering av Anaconda}
\begin{enumerate}
\item Gå til \url{https://www.anaconda.com/download/}
\item Klikk Windows, Mac eller Linux nederst på skjermen.
\item Klikk på Python 3.6 version  DOWNLOAD  (64-Bit).
\item Når filen har lastet ned (typisk 400 MB), kjører du den.
\item Under installasjon er det greieste å si OK til alt.
\end{enumerate}
{\em NB: Under windows kan det oppstå problemer om Anaconda installeres i en path med mellomrom, f.eks. \emph{``Program files''}. Dette skal ikke være et problem om du følger det installasjonen selv foreslår.}

Gjør oppgaver for å lære, ikke bare for å gjøre de. Noen delkapitler har litt for mange oppgaver. Når du føler oppgavene ikke får deg til å lære noe nytt, er det kanskje på tide å gå videre til neste oppgave eller neste underkapittel. 

Mange av delkapitlene er markert med optional. De rekker vi sannsynligvis ikke, men det er alltids tillatt å ta en titt. 

Dette dokumentet gir en grei innføring i de mest grunnleggende delene av python,  samt lett overfladisk beskrivelse av noen ekstrapakker (numpy, matplotlib, pygame). 

Der er alltid (mye) mer å lære om et emne: søk på nettet når du lurer. Særlig stackoverflow.com er bra for å lete etter svar på spørsmål av typen "how to ... in python". 

Noen filer (inkludert denne) er tilgjengelige på \url{http://kilelabs.no/p/}

\section{Hva er programmering?}
Datamaskiner har 2 hovedegenskaper: De er raske, og de er nøyaktige. Det at de også fremstår som ``intelligente'' er en illusjon skapt av programmerere.

Et Google søk er et godt eksempel på dette. Google vet egentlig ingenting om hva du søker etter, men bruker en oppskrift (algoritme) til å gi deg resultater som gir inntrykk av at de vet mer enn de egentlig gjør.

For å få til dette, vedlikeholder Google en stor liste (database) over ord og hvor de finnes. Dette gjør de ved å få datamaskiner til å laste ned nettsider og huske (lagre) hvilke ord som fins på disse sidene. Hver gang datamaskinene finner linker, laster de ned disse nettsidene også og gjør den samme prosessen der. På denne måten dekker de til slutt over mesteparten av alt som fins på Internett.

Hver gang en bruker søker etter ett (eller helst flere) ord, slår Google opp i listen og finner ut hvilke nettsider som inneholder flest av de ordene brukeren søker etter. Siden dette kan dreie seg om tusener eller millioner av nettsider, sorterer den listen basert på hvilke nettsider som oftest blir besøkt av de som søker etter de samme ordene. På denne måten ``lærer'' Google hvordan de kan gi brukeren bedre og bedre resultater etter hvert som flere søker etter og klikker på søkeresultater.

Datamaskinene som er involvert i denne prosessen vet ingenting om hva de prøver å oppnå. De gjør bare akkurat det de får beskjed om fra programmererne. Programmererne ber de, for eksempel, om å laste ned en liste med ord fra en nettside (datamaskinen vet ikke at det dreier seg om sammenhengende tekst). Deretter blir de bedt om å gå gjennom det første ordet, det andre ordet osv og sammenlikne de med listen over ord de har fra før. Når de finner ordet, blir datamaskinen bedt om å huske at den aktuelle nettsiden inneholder dette ordet.

Summen av alle disse små og enkle oppgavene blir til slutt nyttig for brukeren. Det å identifisere og realisere disse oppgavene på en sikker og effektiv måte er jobben til en programmerer.