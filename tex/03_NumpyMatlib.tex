\chapter{Numpy/Matlib}

\section{Motivasjon}

Figur-motivasjon
Følgende program plotter to grafer. 

\begin{usncodebox}
import numpy as np
import matplotlib.pyplot as plt   
x = np.linspace(0,3,31)
y1 = x**2
y2 = 25-x**3
plt.plot(x,y1,'bo-')   # blå, heltrukket linje, sirkel for hvert av punktene  
plt.plot(x,y2,'r--')   # rød, stiplet linje
plt.show()
\end{usncodebox}

Grafen vises inline i IPython-konsollen. Dersom du klikker på tab'en med Python (i stedet for IPython) og kjører på ny, vil grafen bli vist i et eget vindu. (Mer om det senere.) 

Nedenfor ser vi nærmere på numpy og matplotlib, de to pakkene vi bruker når vi lager grafer. 

\section{Numpy: Array}

Numpy står for Numerical Python og inneholder matematiske ressurser som gjør det enklere å bruke Python til numeriske beregninger, plotting etc. 

(Scipy (Scientific Python) er en lignende pakke du fort kan komme borti. Den bygger videre på numpy.)

Spesielt er det datatypen array (ndarray, n-dimensional array) vi trenger fra numpy. (Datatyper vi har vært borti så langt: int, float, str, bool, list, tuple, (dict).) Et array kan lages fra en liste:

\begin{usncodebox}
import numpy
mylist = [1, 2, 3.3, 4.5, 4] 
myarr = numpy.array(mylist) 
type(myarr)    # (numpy.ndarray)
print(myarr)
\end{usncodebox}

(NB: når du printer et array, vises kommaet mellom elementene.)  

Et array er som en spesiell type liste. Du kan endre verdier, f.eks.
\begin{usncodebox}
myarr[1] = 2.44
\end{usncodebox}

Men du kan ikke endre lengden (legge til eller fjerne verdier) ved append(), som vi kan gjøre med en vanlig liste. Skriv myarr. og trykk TAB for å se hvilke funksjoner denne datatypen tilbyr. (Vi skal bruke bare noen få av disse.) 

Andre måter å lage et array på, er:
\begin{usncodebox}
arr1 = numpy.zeros(6)    # lager et array med 6 nuller
arr2 = numpy.ones(6)     # lager et array med 6 enere
\end{usncodebox}

Der finnes to funksjoner for å lage et array som øker jevnt mellom en minste og en største verdi. Den ene: arange() er nesten det range() er for lister: 
\begin{usncodebox}
arr3 = numpy.arange(1, 2, 0.1) 
print(arr3)
\end{usncodebox}

Her lages et array med verdier 1., 1.1, 1.2, ..., 1.9  (NB: maksverdien 2 er ikke med) Til forskjell fra range(), tillater arange() desimalverdier.

Den andre: linspace(a,b,n) lager et array med n verdier. Første verdien er a, siste er b. 
\begin{usncodebox}
arr4 = numpy.linspace(1,2,11)  # gir 1., 1.1, ..., 2.0  
print(arr4)
\end{usncodebox}

Metoden med arange() eller linspace() er praktisk når vi skal lage figurer. Typisk vil vi ønske en jevn sekvens av x-verdier, som er det linspace() og arange() gir. Så regner vi ut y-verdier, en for hver x-verdi, utifra en prosedyre eller formel, og plotter til slutt x mot y. 

Ex.:
\begin{usncodebox}
import numpy
x = numpy.linspace(0,1,11)
y = numpy.zeros(11)
for i in range(11): 
    y[i] = x[i]**2
print(x)
print(y)
\end{usncodebox}

Vi bruker her en loop for å fylle y-arrayet. I for-loopen er i først lik 0 og vi regner ut y[0] basert på x[0], så blir i lik 1 etc., frem til i=10, som er siste elementet i x og y. 

Dette er fint, og vi er nå forsåvidt klare til å bruke matplotlib for å tegne figuren.

Men først: vektorisering
Der finnes imidlertid en raskere og kortere (og dermed bedre) måte å regne ut y på. I tilfellet ovenfor kan loopen erstattes av vektorisering. Vi kan ganske enkelt skrive:
\begin{usncodebox}
y = x**2
\end{usncodebox}

Python vil da se at x er et array og internt gjøre det vi gjør med for-loopen, men på en mer effektiv måte. Vi trenger altså ikke for-loopen vi hadde i eksempelet ovenfor. (NB: dette ville ikke fungert dersom x var en liste av tall (x = [0,.1,.2,.3]), vektorisering fungerer kun med arrays.) 

Programmet blir da:
\begin{usncodebox}
import numpy
x = numpy.linspace(0,1,11)
y = x**2
print(x)
print(y)
\end{usncodebox}

Vi kan også printe ut resultatet slik:
\begin{usncodebox}
for i in range(len(x)):
    print("x:%5.3f  y:%5.3f" %(x[i], y[i]))
\end{usncodebox}

For å tillate vektorisering
Numpy inneholder egne versjoner av forskjellige funksjoner for bruk sammen med array 
der de opprinnelige funksjonene ikke kan brukes til vektorisering. 

F.eks. kan vi skape et array av 10 tilfeldige desimaltall mellom 3 og 5 med numpys 
egen numpy.random.uniform():
\begin{usncodebox} 
arr10 = numpy.random.uniform(3,5,10)
\end{usncodebox}

Dette er ikke mulig med den vanlige random.uniform(). Den tar bare to argumenter, og gir ett tilfeldig tall. Skulle vi laget en liste med ti tilfeldige tall, måtte vi brukt en for-loop.
\begin{usncodebox}
list10 = []
for i in range(10): 
    list10.append( random.uniform(3,5) ) 
\end{usncodebox}

Og dersom vi har et array og vi vil ta kvadratroten av hvert element i arrayet, kan vi gjøre:
\begin{usncodebox}
arr10b = numpy.sqrt(arr10)
\end{usncodebox}

Igjen: Dette er ikke mulig med math sin sqrt-funksjon. 
Prøv: 
\begin{usncodebox}
import math 
arr10c = math.sqrt(arr10).
\end{usncodebox}

Du får feilmelding TypeError. 

Så: Med vektorisering gjør numpy at vi kan utføre matematiske operasjoner som vi ellers måtte brukt en loop for å gjøre. Det er effektivt (raskt), og gir også mer konsis kode. 

NB: Når du printer eller viser et array i spyder, vil alle elementene vises dersom arrayet har ti eller færre elementer. Er det flere enn ti, vil kun de 3 første og de 3 siste elementene vises, med '...' imellom. Prøv:
\begin{usncodebox}
numpy.random.uniform(0,1,10)
numpy.random.uniform(0,1,11)
\end{usncodebox}

\begin{exercise}
Lag et array som går fra 0 til 10 (inkludert endepunktene). Steglengden (spacing) skal være 1  (dvs. [0, 1, 2, ..., 10]). 
\end{exercise}

\begin{exercise}
Lag et array som går fra 0 til 10 (inkludert endepunktene). Steglengden (spacing) skal være 0.2  (dvs. [0, 0.2, 0.4, ..., 9.8, 10])
\end{exercise}

\begin{exercise}
Sammenlign numpy.linspace(0,10,6) med numpy.linspace(0,10,5). Forstå forskjellen
\end{exercise}

\begin{exercise}
Lag et array med 10 tilfeldige desimaltall mellom 0 og 100 
uten å bruke numpy's egne random-funksjoner. 
Dvs. du må bruke random.uniform(0,100).
Gjør så det samme ved hjelp av numpys randomfunksjoner (numpy.random.uniform).
\end{exercise}

\begin{exercise}
En bil starter fra ro (startfart lik 0) og har akselerasjon på 3 m/s2. Lag et program som for hvert sekund fra 0 til 10 sekunder regner ut farten til bilen  og hvor langt den har gått, og skriver det ut til skjerm. Hint: farten v og strekningen s ved tid t er gitt ved v = a*t og s = 0.5*a*t**2 der a = 3 er akselerasjonen.

Utskriften skal se omtrent slik ut:
\begin{tabular}{l|r|l|r|l|r}
tid: & 0.00 s  &  fart: & 0.00 m/s &   strekning: &  0.00 m\\
tid: & 1.00 s  &  fart: & 3.00 m/s &   strekning: &  1.50 m\\
\ldots & & & & & 
\end{tabular}
\end{exercise}

Lese mer: https://en.wikipedia.org/wiki/NumPy

\section{Matplotlib}
\ref{section:matplotlib}

I pakken matplotlib er der mange ressurser som har med plotting å gjøre. Vi skal her bruke en underpakke (sub-package) som heter pyplot.(Den er inspirert av plottingen i Matlab.)

Nedenfor skiller vi mellom (figur)vindu, figur og graf.
\begin{description}
\item{Vindu} Selve figurvinduet, som du kan flytte rundt på skjermen om du vil
\item{Figur} Koordinatsystem
\item{Graf}  Selve linja eller punktene som representerer funksjonen du plotter
\end{description}

Skriv inn følgende program og lagre det som tut\_{}matplotlib1.py. 

\begin{usncodebox}
import numpy 
import matplotlib.pyplot as plt 
x = numpy.linspace(0,1,11)
y = x**2
plt.xlabel('x')   # Vi kan sette titler på aksene
plt.ylabel('y')   # y-aksen
plt.title('En fin funksjon')    # figurtittel 
plt.plot(x,y,'*--') 
  # x er array med x-verdiene
  # y er array med y-verdiene
  # neste argument pynter på grafen: b er blå, * er sirkel, -- er stiplet linje
plt.show()   
\end{usncodebox}

Siste linja er kun nødvendig dersom figuren vises i eget vindu. Viser du grafen inline, (som er det spyder i utgangspunktet gjør), gjør linja ingenting (og skader dermed heller ikke). 

Kjør programmet. Der er to måter figuren kan vises. Enten skjer det inline i IPython-konsollen, som er default. Eller så blir et nytt vindu åpnet ved kommandoen ovenfor. I inline-modus har siste linja, plt.show() ingen effekt. Men er du ikke i inline-modus, og et nytt vindu åpnes, må den linja være med ellers lukker vinduet seg når programmet er ferdig, dvs. så å si umiddelbart. I sistnevnte tilfelle kan vi ikke bruke IPython-terminalen så lenge vinduet er oppe. Avslutt i så fall ved å klikke krysset oppe i høyre hjørne på figur-vinduet. Men du er sannsynligvis i inline-modus. 

For å endre fra/til inline plotting kan du enten ...:
klikke fra toppmenyen: 
Tools / Preferences / IPython console / Graphics
og velg så Backend (Inline eller Automatic) 
Automatic gir et eget figur-vindu.
Du må muligens restarte IPython-konsollen: Consoles / Restart kernel
Det er tar et par sekunder.

... eller du kan skrive i IPython-konsollen en av følgende:
\begin{usncodebox}
%matplotlib inline   # figur inline i IPython-konsollen
%matplotlib qt       # figur i eget vindu
\end{usncodebox}

En annen mulighet for å tvinge figurer til å lages i eget vindu er å bruke Python-konsollen i stedet for IPython-konsollen. 

(Så: dersom du har problem med å få det som du vil, prøv en av de andre mulighetene.) 

Når du plotter i en eget vindu i IPython-terminalen, trenger du plt.show() for å unngå at programmet avsluttes og vinduet forsvinner med en gang. Dette låser imidlertid terminalen, slik at du må klikke vekk figurvinduet for å igjen kunne skrive i terminalen. Dette er ikke alltid det du ønsker. Og der er en løsning. Det er mulig å plotte såkalt interaktivt. Figuren kommer da opp allerede med plt.plot(), og forsvinner ikke når programmet er ferdig, og du trenger ikke plt.show(). Du kan da gjøre endringer. Terminalen låser seg ikke. Det du må gjøre er å bruke plt.ion(), der ion betyr interactive on. (Interaktiv plotting slås av ved plt.ioff()) 

Prøv (i IPython-konsollen):
\begin{usncodebox}
import numpy 
import matplotlib.pyplot as plt 

x = numpy.linspace(0,1,11)
y = x**2

plt.ion()              # interactive on 
plt.plot(x,y,'b*--')   # Figuren vises med en gang, du trenger ikke plt.show()
plt.xlabel('X')        # Kan legge til ting underveis
plt.ylabel('Y') 

Du kan lagre figuren: 
plt.savefig('x_i_andre.png')   # Tillatte formater: png, jpg, pdf, ... 
\end{usncodebox}

Skal du tegne en ny graf, må du lukke figurvinduet eller skrive plt.cla(), ellers blir grafen plottet i den samme figuren. 

\begin{exercise}
Tegn grafen f(x) = x**3-x for -2 < x < 2. Sett passende label på x- og y-aksen, og gi grafen en tittel. Tegn grafen både inline og i eget vindu (i to ulike kjøringer). 
\end{exercise}

Mye mer å lese (matplotlib inneholder mye): 
\url{http://www.labri.fr/perso/nrougier/teaching/matplotlib/}  (tutorial) 
\url{matplotlib.org}
\url{http://matplotlib.org/gallery.html} 
\url{http://matplotlib.org/examples/index.html}
\url{http://matplotlib.org/api/pyplot_summary.html}
\url{http://matplotlib.org/resources/index.html}  (Matplotlib-tutorials etc.)

\section{Mer matplotlib}

Tema: akser, flere subplot i samme vindu, flere vinduer samtidig 

NB: her må plottingen foregår i eget vindu (ikke inline), enten i IPython-konsollen (med ion) eller i Python-vinduet. 

\subsection{Flere grafer i samme figur}

(Starte på nytt ved å klikke vekk vinduet eller skrive plt.cla(), eller ved å restarte IPython-konsollen.) 

\begin{usncodebox}
import numpy                        # om nødvendig
import matplotlib.pyplot as plt     # om nødvendig
#plt.cla()                          # om nødvendig  Fjerner alt fra figuren. (cla: Clear current Axes.) 

x = numpy.linspace(0,2,21)
y1 = x
y2 = x**2                 # vektorisering 
y3 = numpy.sqrt(x)        # bruker igjen vektorisering

# Tre grafer i samme figur. NB: Vi tar med label
plt.plot(x, y1, 'r*-' , label='x')
plt.plot(x, y2, 'g^--', label='x**2')
plt.plot(x, y3, 'b-.' , label='sqrt(x)') 

plt.legend()    # Først her blir label'ene vist 

plt.title('Noen grafer')  # Tittel
plt.xlabel('X') 
plt.ylabel('Y')

# Vi kan bestemme hvor mye av x- og y-aksen som skal plottes [xmin,xmax, ymin,ymax]
plt.axis([0,2, -1,2])     # x mellom 0 og 2, y mellom -1 og 2  (ikke optimalt)
plt.axis('tight')         # plotter akkurat så mye som trengs for å vise grafene 
                          # (er default dersom axis ikke blir satt)
 
plt.savefig('noenfigurer.jpg')  # lagrer figuren, kan velge format (jpg, png, pdf, eps, ...). 
\end{usncodebox}

Det går også an å lagre figuren direkte fra figurvinduet ved å klikke diskettikonet. 

\subsection{Flere figurer i samme figur-vindu: Subplot}

subplot(k, r, i) tar tre argumenter
\begin{itemize}
\item K: antall kolonner 
\item R: antall rader
\item i: hvilket subplot som gjøres til det aktive (1: oppe til venstre, 2: en til høyre, etc. Når du har nådd antall kolonner, går du en rad ned)
\end{itemize}
subplot() setter altså av plass til K x R subplott, og velger det aktive.

\begin{usncodebox}
plt.subplot(2,2,1)          # oppe til venstre 
plt.plot(x,y1,'*--b')       # 
plt.title("x")

plt.subplot(2,2,2)          # oppe til høyre
plt.plot(x,y2,'*--g')       # 
plt.title("x**2")

plt.subplot(2,2,3)          # nede til venstre
plt.plot(x,x**3,'*--y')     # 
plt.title("x**3")

plt.subplot(2,2,4)          # nede til høyre
plt.plot(x,x**0.25,'*--r')  # 
plt.title('x**0.25')
\end{usncodebox}

Det går an å endre størrelsen på figuren-vinduet manuelt. 

Hvordan fjerne grafene ovenfor? Å klikke vekk vinduet er en mulighet. Da er vi tilbake til 1-figur-i-vinduet neste gang vi plotter. 

plt.cla()  fungerer ikke. Det bare fjerner den aktive figuren, som for oss figuren nede til høyre. 

plt.subplot(1,1,1) fungerer. Da sier vi at vi skal ha 1 figur i vinduet, og alle 4 blir fjernet 
og vi står med blanke ark. 

\subsection{Flere figurvinduer}

Eksempel:
\begin{usncodebox}
plt.figure("fig1")                     # lag et figurvindu (og la det være det aktive)
plt.plot(x, 3+x**2, '*--b')          # legg en graf i figurvinduet
plt.plot(x, 3-x**2, '*--r')          # legg en graf til i figurvinduet

plt.figure("fig2")                     # lag et annet figurvindu
plt.plot(x, x**1.5,'*--g')             # legg en graf i dette vinduet (siden det nå er det aktive vinduet)

plt.figure("fig3")                     # skift tilbake til det første vinduet (fig1 blir nå det aktive vinduet)
plt.plot(x, 2+x**4,'*--y')             # legg en graf til i det aktive vinduet (som er fig1)
\end{usncodebox}

\begin{exercise}
Lag et program som åpner et figur-vindu med fire subfigurer. Utgangspunktet er skrått kast med starthastighet 10 m/s og både x- og y-retning (som vi hadde tidligere). I fire subfigurvinduer skal så fart i x- og y-retning (vx og vy) plottes, og tilbakelagt strekning i x- og y-retning (sx og sy). Hint: vx = vx0, vy = vy0`+ g*t, sx = vx*t, sy = vy0*t + 0.5*a*t**2  der a=-9.81 (tyngdeakselerasjonen) (Diskuter med sidemannen.) 
\end{exercise}

Mange eksempler som viser hva man kan gjøre med matplotlib
https://matplotlib.org/examples/index.html

\section{Scatterplott (optional)}

Scatterplot:
I figuren ovenfor økte x-verdien med indeksen (x[0] < x[1] < x[2] < ... ). Ofte er det det naturlige, f.eks. tid på x-aksen og posisjon på y-aksen. I et scatterplott trenger ikke x-verdiene å øke, de kan være tilfeldige. Man er kun interessert i xy-parene. F.eks. la x være lengden på en bil og y være tiden det tar å akselerere til 100 km/t.

\begin{usncodebox}
import numpy 
import matplotlib.pyplot as plt

N = 7
x = numpy.random.uniform(3.5,6,N)   # N tilfeldige billengder (alle mellom 3.5 og 6 meter)
y = numpy.random.uniform(4,12,N)    # N tilfeldige 0-100-tider (alle mellom 4 og 12 sekunder)
#plt.figure("scatter: ordinary")    # åpner nytt figurvindu  (kan droppe)
plt.cla()                           # rensker figurvinduet (dersom du allerede har et figurvindu), evt. åpner et nytt
plt.scatter(x, y)                   # scatterplot: x vs y
plt.xlabel('billengde')
plt.ylabel('tid 0-100 km/t')
\end{usncodebox}

Med et enkelt scatterplot som det ovenfor, ville plt.plot(x,y,'o') gitt samme plott. Merk at vi ikke er interessert i linje mellom punktene, siden der ikke er noen sammenheng/kronologi mellom x-verdiene. 

Scatterplot har noen grafiske finesser som kan være vakre og/eller nyttige. Du kan angi ulik størrelse per punkt, og farge, samt gjennomsiktighet: 

\begin{usncodebox}
import numpy 
import matplotlib.pyplot as plt

N = 50   # nå 50 biler 
x = numpy.random.uniform(3.5,6,N)   # N tilfeldige billengder (alle mellom 3.5 og 6 meter)
y = numpy.random.uniform(4,12,N)    # N tilfeldige 0-100-tider (alle mellom 4 og 12 sekunder)
colors = numpy.random.rand(N)       # tilfeldig farge til hvert punkt
areal = numpy.pi * (15 * numpy.random.rand(N))**2    # 0-15 punkts radius 
plt.scatter(x, y, s=areal, c=colors, alpha=0.5)

# colors: farge angitt med float mellom 0 og 1.
# Fargene følger et fargekart med lilla nederst (0), via blå og grønn mot gul (1) (kan endres). 
# area: størrelse på punktene, bruker areal = pi*radius**2
# alpha: gjennomsiktighet, 0:helt gjennomsiktig, 1:ingen gjennomsiktighet
\end{usncodebox}

\section{Animasjon i for-løkkke (optional)}

Ting som beveger seg er gøy, og kan også være informativt. Det er mulig å lage ting som beveger seg via matplotlib. Det kan være funksjoner som oppdaterer seg ettersom tiden går, f.eks. sin(t). Det kan være (veldig) enkle fysikkanimasjoner, som f.eks. en ball som blir kastet. Animasjonen kan vises mens programmet kjører, eller man kan også lage en liten video-fil (mp4) som f.eks. kan legges ut på nettet. Det kan imidlertid også bli litt vel teknisk på dette nivået. 

En enkel, men begrenset og vanligvis hakkete måte er å oppdatere grafen i en loop. Det fungerer ikke dersom figurene vises inline. Man bruker da ingen av de mer avanserte animasjons-funksjonene i matplotlib, og kan heller ikke lage mp4-fil.

Eksempel (skrått kast):
(Skriv inn og lagre som tut\_{}skraattkast\_{}anim\_{}for.py) 

\begin{usncodebox}
import numpy as np
from matplotlib import pyplot as plt

plt.ion()   # interaktiv kjøring
fig = plt.figure()    # åpner et figurvindu
akse = plt.axes(xlim=(0, 25), ylim=(0, 10))  # setter opp koordinatsystem

# initialisere kastet
vx0 = 10       # startfart i x-retning
vy0 = 10       # startfart i y-retning

dt = 0.1       # oppdaterer hvert 0.1 sekund
t = 0          # starttid
ay = -9.81     # akselerasjon i y-retning er tyngdeakselerasjonen (nedover, derfor minus)
x = 0 
y = 0          # startposisjon 

while y >= 0:  # stopper når ballen når bakken
    x = vx0*t
    y = vy0*t + 0.5*ay*t**2
    akse.scatter([x],[y],c='r')  # her plottes den nye posisjonen (den forrige slettes ikke)
    t = t + dt          # oppdatere tiden
    plt.pause(0.02)     # liten pause for aa gjoere plottingen litt realistisk
\end{usncodebox}

Dette fungerer forsåvidt, men det kan altså gjøres bedre.

\section{Animasjon: avansert, lage mp4 (optional)}

Nedenfor gjøres det samme. Her bruker vi de mer avanserte funksjonene. NB: kan måtte trenge å installere ffmpeg eller mencoder. 

Legg følgende kode inn i en fil, og kall den gjerne tut\_{}animasjon\_{}skraattkast\_{}mp4.py. (Ligger på http://kilelabs.no/p/tut/)

\begin{usncodebox}
import numpy as np
from matplotlib import pyplot as plt
from matplotlib import animation

plt.ion()   # ved interaktiv kjoering 

# Sett opp figur og plotteelementene vi ønsker å animere
fig = plt.figure()                           # aapner et figurvindu
akse = plt.axes(xlim=(0, 25), ylim=(0, 10))  # setter opp koordinatsystem

# Initialiser fysikken
x0  = 0
y0  = 0
vx0 = 10
vy0 = 10
t0  = 0      # starttid
ay  = -9.81  # akselerasjon i y-retning er tyngdeakselerasjonen (nedover, derfor minus)
x   = 0
y   = 0

p = akse.scatter([x],[y],c='r')
akse.xaxis.set_label_text('lengde (m)')  # optional
akse.yaxis.set_label_text('høyde (m)')   # optional

# 
fps = 30      # vil ha 30 frames per second (i mp4-fil)
dt = 1./fps   # tid mellom hver oppdatering 

t_tot = 2*vy0/(-ay) + 1      # la programmet regne ut totaltid (og legge til 1 sek)
#t_tot = 3                   # eller for hand: totaltid er ca. 2 sekunder. Legger til 1
n_frames = int(fps * t_tot)  # antall frames (heltall). Trenger dette nedenfor. 



# Til animasjonen trenger vi to funksjoner: animate() og init() 

def init():
    p = akse.scatter([x0],[y0])
    return p,


def animate(i):
    # Her oppdateres tid og posisjon, og grafen
    t = t0 + i*dt
    x = vx0*t
    y = vy0*t + 0.5*ay*t**2
    p = akse.scatter([x],[y],c='r')
    return p,
    

# Saa kaller vi matplotlib sin animator-funsjon.  
anim = animation.FuncAnimation(fig, animate, init_func=init, frames=n_frames, interval=dt*1000, blit=True)
# fig       : er den plt.figure vi bruker
# animate   : er funksjonen vi har definert som regner ut neste steg (og tegner)
# init_func : startsituasjonen
# frames    : antall frames i animasjonen (i går fra 0 til antall frames)
# interval  : tid i ms mellom hver frame 
# blit=True : kun det som har endret seg blir tegnet 


# Til slutt lagres animasjonen som mp4 (kommentert ut i foerste omgang)
#anim.save('tut_scatter_skraattkast.mp4', fps=fps, extra_args=['-vcodec', 'libx264'])
# Dette krever at en encoder er installert (ffmpeg, mencoder, ...)
# Argumentet libx264 gjør at videoen kan vises i html5. 
# Detaljer:  http://matplotlib.sourceforge.net/api/animation_api.html
\end{usncodebox}

Prøv å kjøre programmet. Skal vise animasjon av skrått kast. 

Prøv også å lage mp4, dvs. fjerne '\#' fra anim.save-linja (mot slutten). Du får sannsynligvis feilmelding om at ffmpeg el.l. ikke er installert. 

Vi skal ikke gå i mer detalj her. For å lære mer om hva som kan gjøres i matplotlib og hvordan, sjekk ut matplotlib-referansene gitt i av \cite{section:matplotlib}

\section{Animasjon : avansert : ballen spretter tilbake  (optional)}

Skrått kast igjen, men nå spretter ballen opp igjen fra bakken, og tilbake fra veggene. Se http://kilelabs.no/p/tut/
tut\_{}scatter\_{}ball\_{}bouncing.py
tut\_{}scatter\_{}ball\_{}bouncing.mp4

\section{Animasjon : avansert, lage mp4 (optional)}

Nedenfor gjøres det samme. Her bruker vi de mer avanserte funksjonene. NB: kan måtte trenge å installere ffmpeg eller mencoder. 

Legg følgende kode inn i en fil, og kall den gjerne tut\_{}animasjon\_{}skraattkast\_{}mp4.py. (Ligger på http://kilelabs.no/p/tut/)

\begin{usncodebox}
import numpy as np
from matplotlib import pyplot as plt
from matplotlib import animation

plt.ion()   # ved interaktiv kjoering 

# Sett opp figur og plotteelementene vi ønsker å animere
fig = plt.figure()                           # aapner et figurvindu
akse = plt.axes(xlim=(0, 25), ylim=(0, 10))  # setter opp koordinatsystem

# Initialiser fysikken
x0  = 0
y0  = 0
vx0 = 10
vy0 = 10
t0  = 0      # starttid
ay  = -9.81  # akselerasjon i y-retning er tyngdeakselerasjonen (nedover, derfor minus)
x   = 0
y   = 0

p = akse.scatter([x],[y],c='r')
akse.xaxis.set_label_text('lengde (m)')  # optional
akse.yaxis.set_label_text('høyde (m)')   # optional

# 
fps = 30      # vil ha 30 frames per second (i mp4-fil)
dt = 1./fps   # tid mellom hver oppdatering 

t_tot = 2*vy0/(-ay) + 1      # la programmet regne ut totaltid (og legge til 1 sek)
#t_tot = 3                   # eller for hand: totaltid er ca. 2 sekunder. Legger til 1
n_frames = int(fps * t_tot)  # antall frames (heltall). Trenger dette nedenfor. 



# Til animasjonen trenger vi to funksjoner: animate() og init() 

def init():
    p = akse.scatter([x0],[y0])
    return p,


def animate(i):
    # Her oppdateres tid og posisjon, og grafen
    t = t0 + i*dt
    x = vx0*t
    y = vy0*t + 0.5*ay*t**2
    p = akse.scatter([x],[y],c='r')
    return p,
    

# Saa kaller vi matplotlib sin animator-funsjon.  
anim = animation.FuncAnimation(fig, animate, init_func=init, frames=n_frames, interval=dt*1000, blit=True)
# fig       : er den plt.figure vi bruker
# animate   : er funksjonen vi har definert som regner ut neste steg (og tegner)
# init_func : startsituasjonen
# frames    : antall frames i animasjonen (i går fra 0 til antall frames)
# interval  : tid i ms mellom hver frame 
# blit=True : kun det som har endret seg blir tegnet 


# Til slutt lagres animasjonen som mp4 (kommentert ut i foerste omgang)
#anim.save('tut_scatter_skraattkast.mp4', fps=fps, extra_args=['-vcodec', 'libx264'])
# Dette krever at en encoder er installert (ffmpeg, mencoder, ...)
# Argumentet libx264 gjør at videoen kan vises i html5. 
# Detaljer:  http://matplotlib.sourceforge.net/api/animation_api.html
\end{usncodebox}

Prøv å kjøre programmet. Skal vise animasjon av skrått kast. 

Prøv også å lage mp4, dvs. fjerne '\#' fra anim.save-linja (mot slutten).Du får sannsynligvis feilmelding om at ffmpeg el.l. ikke er installert. 

Vi skal ikke gå i mer detalj her. For å lære mer om hva som kan gjøres i matplotlib og hvordan, sjekk ut matplotlib-referansene gitt i av \cite{section:matplotlib}. 

\section{Animasjon : avansert : ballen spretter tilbake  (optional)}

Skrått kast igjen, men nå spretter ballen opp igjen fra bakken, og tilbake fra veggene.
Se http://kilelabs.no/p/tut/
tut\_{}scatter\_{}ball\_{}bouncing.py 
tut\_{}scatter\_{}ball\_{}bouncing.mp4
