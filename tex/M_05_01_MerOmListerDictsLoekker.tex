######################################################################## 
KAP. 5.1  MER OM LISTER, DICTS OG LØKKER

Å kopiere en liste:

L1 = ['a','c','b']
L2 = L1
Dette lager ikke en kopi av liste L1.
L2 peker bare på samme objekt i hukommelsen.
Så dersom den ene endres, endres også den andre, det er samme objekt.
Se: 
print(L2)       # L2 er ['a','c','b']
L1.append('d')  # Vi endrer L1
print(L2)       # L2 har blitt ['a','c','b','d']

Dersom vi vil lage en frittstående kopi, kan vi bruke 
L3 = L1.copy()
(eller L3 = list(L1))



Å kopiere en dictionary: 

Hvis du vil lage en frittstående kopi av en dictionary, må du bruke copy()
D1 = {'en':'one', 'to':'two'}
D2 = D1           # peker på samme objekt
D3 = D1.copy()    # lager en nytt, frittstående dictionary



To måter å gå gjennom en liste på: 
Via indeks: 
for i in range(len(L1)):
    print("%i  %s" %(i,L1[i]))
Eller bare via verdiene: 
for L in L1:
    print(L) 


Sortere en liste (ordne alfabetisk m.m.)
sorted(L1)               # endrer ikke L1, men gir en sortert liste 
L1s = sorted(L1)         # 
L1.sort()                # sorterer (endrer) L1
L1.sort(reverse=True)    # alfabetisk baklengs
sorted(L1,reverse=True)  # endrer ikke L1



Løkker: break og continue 

Continue brukes for å droppe resten av linjene og gå direkte til neste 
iterasjon i for- eller while-loopen
Eksempel: 
for i in range(5): 
    print(i, 'before')
    if i == 2: continue
    print(i, 'after')

Break brukes for å hoppe helt ut av en loop: 
Eksempel 1: 
for i in range(5): 
    print(i)
    if i == 2: break

Eksempel 2
for i in range(3): 
    for j in range(5): 
         if j == 2: break  # hopper ut av j-loopen
         print(i,j)



