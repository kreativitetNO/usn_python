######################################################################## 
KAP. 3.7  ANIMASJON : avansert, lage mp4 (optional)

Nedenfor gjøres det samme.
Her bruker vi de mer avanserte funksjonene. 
NB: kan måtte trenge å installere ffmpeg eller mencoder. 

Legg følgende kode inn i en fil, og kall den gjerne tut_animasjon_skraattkast_mp4.py. 
(Ligger på http://kilelabs.no/p/tut/)


import numpy as np
from matplotlib import pyplot as plt
from matplotlib import animation

plt.ion()   # ved interaktiv kjoering 

# Sett opp figur og plotteelementene vi ønsker å animere
fig = plt.figure()                           # aapner et figurvindu
akse = plt.axes(xlim=(0, 25), ylim=(0, 10))  # setter opp koordinatsystem

# Initialiser fysikken
x0  = 0
y0  = 0
vx0 = 10
vy0 = 10
t0  = 0      # starttid
ay  = -9.81  # akselerasjon i y-retning er tyngdeakselerasjonen (nedover, derfor minus)
x   = 0
y   = 0

p = akse.scatter([x],[y],c='r')
akse.xaxis.set_label_text('lengde (m)')  # optional
akse.yaxis.set_label_text('høyde (m)')   # optional

# 
fps = 30      # vil ha 30 frames per second (i mp4-fil)
dt = 1./fps   # tid mellom hver oppdatering 

t_tot = 2*vy0/(-ay) + 1      # la programmet regne ut totaltid (og legge til 1 sek)
#t_tot = 3                   # eller for hand: totaltid er ca. 2 sekunder. Legger til 1
n_frames = int(fps * t_tot)  # antall frames (heltall). Trenger dette nedenfor. 



# Til animasjonen trenger vi to funksjoner: animate() og init() 

def init():
    p = akse.scatter([x0],[y0])
    return p,


def animate(i):
    # Her oppdateres tid og posisjon, og grafen
    t = t0 + i*dt
    x = vx0*t
    y = vy0*t + 0.5*ay*t**2
    p = akse.scatter([x],[y],c='r')
    return p,
    

# Saa kaller vi matplotlib sin animator-funsjon.  
anim = animation.FuncAnimation(fig, animate, init_func=init, frames=n_frames, interval=dt*1000, blit=True)
# fig       : er den plt.figure vi bruker
# animate   : er funksjonen vi har definert som regner ut neste steg (og tegner)
# init_func : startsituasjonen
# frames    : antall frames i animasjonen (i går fra 0 til antall frames)
# interval  : tid i ms mellom hver frame 
# blit=True : kun det som har endret seg blir tegnet 


# Til slutt lagres animasjonen som mp4 (kommentert ut i foerste omgang)
#anim.save('tut_scatter_skraattkast.mp4', fps=fps, extra_args=['-vcodec', 'libx264'])
# Dette krever at en encoder er installert (ffmpeg, mencoder, ...)
# Argumentet libx264 gjør at videoen kan vises i html5. 
# Detaljer:  http://matplotlib.sourceforge.net/api/animation_api.html




Prøv å kjøre programmet. 
Skal vise animasjon av skrått kast. 

Prøv også å lage mp4, dvs. fjerne '#' fra anim.save-linja (mot slutten).
Du får sannsynligvis feilmelding om at ffmpeg el.l. ikke er installert. 

Vi skal ikke gå i mer detalj her.
For å lære mer om hva som kan gjøres i matplotlib og hvordan,
sjekk ut matplotlib-referansene gitt i av KAP. 3.3. 
