% ------------------------------------------------------------------------
% file `document-exercise-33-exercise-body.tex'
%
%     exercise of type `exercise' with id `33'
%
% generated by the `exercise' environment of the
%   `xsim' package v0.11 (2018/02/12)
% from source `document' on 2018/08/08 on line 1168
% ------------------------------------------------------------------------
Lag et gangetabelltestingsprogram som først tar følgende input fra brukeren:
\begin{itemize}
\item Antall oppgaver som må klares (f.eks. 10)
\item Hvor mange minuspoeng man får av å svare feil (f.eks. 1)
\item Det laveste og høyeste tallet i gangetabellen som man blir spurt fra, f.eks. 5 og 12.
\end{itemize}
Da er det høyeste gangestykket 12x12, det laveste 5x5, og alt imellom er tillatt. (Så f.eks. 5x6, 5x12, 11x5, 12x5 er tillatt, men ikke f.eks. 3x11.) Deretter, i en passende loop, vil to tilfeldige tall (a og b) bli trukket. Du vil så bli spurt om produktet (a ganger b). Svarer du rett, får du poeng, svarer du feil mister du poeng. Når du har svart korrekt det bestemte antall ganger, går programmet ut av loopen. Programmet skriver så ut hvor mange forsøk du trengte. (La programmet gi en optimistisk kommentar til brukeren.)
