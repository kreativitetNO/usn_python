% ------------------------------------------------------------------------
% file `document-exercise-65-exercise-body.tex'
%
%     exercise of type `exercise' with id `65'
%
% generated by the `exercise' environment of the
%   `xsim' package v0.11 (2018/02/12)
% from source `document' on 2018/08/08 on line 100
% ------------------------------------------------------------------------
En bil starter fra ro med akselerasjon lik a = 5 m/s**2. Etter en tid t har bilen farten v = a*t. Implementer farten som en funksjon f3. Bruk trapes-regelen til å finne strekningen bilen har kjørt i løpet av 10 sekunder. Sammenlign dette med den eksakte løsningen gitt ved s = 0.5*a*t**2. Hvor stor n trenger du for å ha en relativ feil mindre enn 0.00001 ?
